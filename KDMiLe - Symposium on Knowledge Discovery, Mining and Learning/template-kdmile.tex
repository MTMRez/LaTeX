% by Mirella M. Moro; version: January/18/2012 @ 04:16pm
% -- 01/18/2012: more discussion on SBBD + kdmile; overall revision
% -- 09/03/2010: bib file with names for proceedings and journals; cls with shrinked {received}
% -- 08/27/2010: appendix, table example, more explanation within comments, editors' data

\documentclass[kdmile,a4paper]{kdmile} % NOTE: kdmile is published on A4 paper
\usepackage{graphicx,url, xcolor }  % for using figures and url format
\usepackage[T1]{fontenc}   % avoids warnings such as "LaTeX Font Warning: Font shape 'OMS/cmtt/m/n' undefined"


%\usepackage{cite} % NOTE: do **not** include this package because it conflicts with kdmile.bst

% Standard definitions
\newtheorem{theorem}{Theorem}[section]
\newtheorem{conjecture}[theorem]{Conjecture}
\newtheorem{corollary}[theorem]{Corollary}
\newtheorem{proposition}[theorem]{Proposition}
\newtheorem{lemma}[theorem]{Lemma}
\newdef{definition}[theorem]{Definition}
\newdef{remark}[theorem]{Remark}

% New environment definition
\newenvironment{latexcode}
{\ttfamily\vspace{0.1in}\setlength{\parindent}{18pt}}
{\vspace{0.1in}}

% ALL FIELDS UNTIL BEGIN{document} ARE MANDATORY


% Includes headers with simplified name of the authors and article title
\markboth{Thiago H. C. Oliveira, Cristiane N. Nobre, Mark A. J. Song, Luis E. Z. Galvez}
{Triadic Rules for the implementation of Activity-based Working Environments - An analysis of the Productive and Social impacts}
%  -> \markboth{}{}
%         takes 2 arguments
%         ex: \markboth{A.Plastino}{Any article title}


% Title of the article
\title{{Triadic Rules for the implementation of \textit{Activity-based Working Environments} - An analysis of the Productive and Social impacts}}


% List of authors
%IF THERE ARE TWO or more institutions, please use:
%\author{Name of Author1\inst{1}, Name of Author2\inst{2}, Name of Author3\inst{2}}
\author{Thiago H. C. Oliveira, Mark A. J. Song, Luis E. Zárate}


%Affiliation and email
\institute{Pontifícia Universidade Católica de Minas Gerais, Departamento de Ciência da Computação, Brasil \\
% IF THERE IS ANOTHER INSTITUTION:
\email{thcoliveira@sga.pucminas.br, song@pucminas.br, zarate@pucminas.br}
}


% Article abstract - it should be from 100 to 300 words
\begin{abstract}
 A longitudinal database records data and its variations over a period of time. The objective of this article is to use this resource, together with the Triadic Concept Analysis theory, to analyze and characterize how employees adapted and felt before, during and after the implementation of an activity-based work environment which is defined as a flexible work setting where employees have the autonomy to choose where they perform their tasks, seeking locations that offer optimal solutions in terms of social interaction, communication, and collaboration. The results seek to support the implementation of this concept, verifying how, and under what conditions, key points of employee experiences vary over time.
\end{abstract}

%% ACM Computing Classification System categories
%% The code below is generated by the tool at http://dl.acm.org/ccs.cfm.
%% Please copy and paste the code instead of the example below.
%%
\begin{CCSXML}
	<ccs2012>
	<concept>
	<concept_id>10010147.10010257.10010321</concept_id>
	<concept_desc>Computing methodologies~Machine learning algorithms</concept_desc>
	<concept_significance>500</concept_significance>
	</concept>
	</ccs2012>
\end{CCSXML}

\ccsdesc[500]{Computing methodologies~Machine learning algorithms}


% Article keywords
\keywords{ Activity Based Working Environments, Data mining, Longitudinal Data Mining, Triadic Concept Analysis, Triadic Rules}
%  -> \keywords{} (in alphabetical order \keywords{document processing, sequences,
%                      string searching, subsequences, substrings})



% THE ARTICLE BEGINS
\begin{document}

\maketitle


% ARTICLE NEW SECTION
\section{Introdução}

Bases de dados longitudinais correspondem a estudos nos quais são observadas as variações de uma mesma amostra de objetos ou indivíduos ao longo de um período de tempo consecutivo pré-determinado chamados de ondas. Essas bases de dados podem ser aplicadas em diversas áreas, tais como saúde, ecologia, estudos sociais, entre outras. Por exemplo, no âmbito social, esses estudos permitem a análise do comportamento de um indivíduo antes, durante e após uma determinada mudança comportamental, seja ela relacionada às pessoas em si ou ao ambiente no qual elas interagem. Por desempenhar um papel essencial na verificação e reconhecimento de padrões no objeto de estudo, torna-se uma ferramenta crucial na validação e descrição dos procedimentos empregados em um ambiente de trabalho.


Um Ambiente de Trabalho Orientado a Atividades (\textit{Activity-based Working Environments} - ABW) é caracterizado por ser um ambiente de trabalho flexível no qual os funcionários têm autonomia para escolher onde desempenhar suas tarefas, buscando locais que ofereçam as melhores soluções relacionadas ao aspecto social, comunicativo e colaborativo.

Com o avanço constante da tecnologia da informação e a crescente conectividade dos ambientes de trabalho, tanto internamente quanto externamente, o conceito de ABW tem se destacado, uma vez que visa  otimizar os espaços úteis, o aumento do bem-estar dos funcionários, o incremento da produtividade, bem como fomentar  o intercâmbio de informações.

Desta forma, o presente trabalho se baseia no Estudo Longitudinal sobre a implementação de um ambiente de trabalho baseado em atividades \cite{HALLDORSSON2022107920} para extrair informações relacionadas à flutuação das métricas de satisfação e produtividade: a) antes da implantação de ABW, b) durante a implantação de ABW; e c) após a implantação de ABW; além de estabelecer a relação entre as métricas nos seus respectivos espaços de tempo. O objetivo é auxiliar a tomada de decisão quanto a adoção desse tipo de ambiente de trabalho em um escritório. O estudo longitudinal contém 11 atributos e 100 registros, acompanhando 100 funcionários antes, durante e após a aplicação do conceito de \textit{Activity-based Working Environments} em seu local de trabalho.

Para a extração das informações utiliza-se da Análise Formal de Conceitos (AFC), que é um ramo da matemática aplicada  relacionada com a hierarquização de conceitos a partir de um conjunto de objetos, atributos e propriedades \cite{ganter1999formal}. Pode-se aplicar uma extensão da AFC, denominada Análise Triádica de Conceitos (ATC) que permite associar a uma certa informação (implicação) uma condição, identificando assim causa-efeito para mudanças em variáveis ao longo do tempo. Também é possível caracterizá-las quanto a sua eficácia, positiva ou negativa, em um certo contexto. Por ser ainda pouco explorada, se torna foco deste trabalho.

Neste estudo, realiza-se a extração de regras triádicas com o intuito de caracterizar a eficiência e a satisfação do escritório ao longo do tempo. Os resultados obtidos destacam o potencial da Análise Triádica de Conceitos \cite{inproceedings} na investigação de bases de dados longitudinais. De maneira geral, a observação da implementação de um Ambiente de Trabalho Orientado a Atividades (ABW) por meio de um estudo longitudinal possibilita a compreensão dos impactos dessa transição tanto no ambiente de trabalho em si quanto nos funcionários envolvidos.

Este artigo está assim organizado da seguinte forma: na Seção 2, encontra-se o referencial teórico que abrange a Análise Formal de Conceitos e a Análise de Dados Longitudinais. Na Seção 3, são apresentados os trabalhos relacionados. A Seção 4 descreve a metodologia adotada, incluindo os materiais e métodos utilizados. A Seção 5 apresenta os resultados e discussões baseados nas métricas de avaliação. Por fim, a Seção 6 traz as considerações finais, concluindo com possíveis trabalhos futuros.


 \section{Referência Teórica}

 \subsection{Análise Triádica de Conceitos}

 A Análise Formal de Conceitos (AFC) é um ramo da matemática aplicada diretamente relacionada com a hierarquização de conceitos, a partir de um conjunto de objetos, atributos e relações de incidência entre estes \cite{ganter1999formal} com o objetivo de identificar propriedades relevantes dos dados.

 Na AFC, um contexto formal corresponde a uma tupla de forma $K := (G, M, I)$, em que G é um conjunto de objetos (extensão), M um conjunto de atributos (intensão), e I uma relação de incidência $(I \subseteq G\times M)$, indicando a incidência de determinados atributos em objetos. Regras de implicação e conceitos formais podem ser extraídos de contextos diádicos. A tabela ~\ref{diadictable} representa um contexto diádico.

\begin{table}[h]
\centering
\caption{Contexto Diádico representado por uma tabela cruzada}
\label{diadictable}
\begin{tabular}{|c|c|l|l|c|l|l|c|l|l|}
\hline
\textit{\textbf{G}}/\textit{\textbf{M}} & \multicolumn{3}{c|}{\textbf{$a_{1}$}} & \multicolumn{3}{c|}{\textbf{$a_{2}$}} & \multicolumn{3}{c|}{\textbf{$a_{3}$}} \\ \hline
\textbf{\textit{$o_{1}$}}  & \multicolumn{3}{c|}{$\times$}  & \multicolumn{3}{c|}{$\times$}  & \multicolumn{3}{c|}{}   \\ \hline
\textbf{\textit{$o_{2}$}}  & \multicolumn{3}{c|}{}   & \multicolumn{3}{c|}{$\times$}  & \multicolumn{3}{c|}{$\times$}   \\ \hline
\textbf{\textit{$o_{3}$}}  & \multicolumn{3}{c|}{$\times$}  & \multicolumn{3}{c|}{}   & \multicolumn{3}{c|}{}   \\ \hline
\end{tabular}
\end{table}
 
 Com base no contexto formal, é possível obter conceitos formais. Um conceito formal de um contexto formal \textit{(G, M, I)} é definido por um par \textit{(A, B)} onde $ \textit{A} \subseteq $ \textit{G}, $ \textit{B} \subseteq $ \textit{M}.  O par \textit{(A, B)} segue a seguinte condição: \textit{A} $=$ \textit{B'} e \textit{B} $=$ \textit{A'} - a operação ( ' )  é definida como derivação e é dada por:
 
 \begin{enumerate}
     \item $A' = \{m \in M | (g, m) \subseteq I$ $ \forall g \in A\}$
     \item $B' = \{g \in G | (g, m) \subseteq I $ $\forall m \in B\}$
 \end{enumerate}

 Nesta relação a extensão \textit{A} contém cada objeto de \textit{G} que possui todos os atributos de \textit{B}, e a intensão \textit{B} contém todos atributos de \textit{M} pertencentes a todos objetos de \textit{A}.

Também é possível extrair regras de implicação que possuem a forma $P \rightarrow Q$, onde $P$ e $Q$ são conjuntos de atributos e $P' \subseteq Q'$, siginifica que um dado objeto que possui $P$, também possui $Q$. Uma regra de implicação \textit{P} $\rightarrow$ \textit{Q} é  válida, se e somente se, todo objeto que possuir os atributos de \textit{P} também possuir os atributos de \textit{Q}. As implicações são dependências entre elementos de um conjunto obtidos de um contexto formal.


A cada regra pode-se associar métricas de avaliação, como o de suporte e a confiança. Fomalmente, o Suporte corresponde à proporção de objetos no subconjunto $ g \in G$ que satisfazem a implicação \textit {$ P \rightarrow Q $}, em relação ao número total de objetos $ | G | $ do contexto formal $ K $ (Equação \ref {suporte}) onde $ (' )$ corresponde ao operador de derivação.



\begin{equation}
\label{suporte}
 Suporte(P \to Q) = \frac{|(P \cup \{Q\})'|}{|G|}
\end{equation}

A confiança corresponde à proporção de objetos $ g \in G $ que contêm $ P $, que também contém $ Q $, para o número total de objetos $ | G | $ (Equação \ref {confianca}).

\begin{equation}
 Conf(P \to Q) = \frac{|(P \cup \{Q\})'|}{|P'|} =  \frac{Suporte(P \to Q)}{Suporte(P)} \label{confianca}
\end{equation}

\subsection{Análise Formal de Conceitos Triádicos (TCA)}
A TCA estende a teoria da AFC clássica com a inserção de uma nova dimensão, pois em algumas análises torna-se essencial associar uma condição. Neste caso, o contexto formal passa a ser definido por uma quádrupla $K = (K1, K2, K3, Y)$, em que K1, K2 e K3 são respectivamente conjuntos de objetos, atributos e condições, e Y corresponde a uma relação ternária entre os mesmos ($I \subseteq K1 \times K2 \times K3 $). A tabela ~\ref{triadictable} representa um contexto triádico.

Apesar de ser oriunda da AFC, a abordagem triádica possui definições de conceito, regras de implicações e derivação, bem mais complexas que a abordagem diádica. Por exemplo, um conceito formal triádico é definido agora por uma tripla ($A_{1}$, $A_{2}$, $A_{3}$), tal que $A_{1}$ $\subseteq$ $K_{1}$, $A_{2}$ $\subseteq$ $K_{2}$ e $A_{3}$ $\subseteq$ $K_{3}$ e $A_{1}$ $\times$ $A_{2}$ $\times$ $A_{3}$ $\subseteq$ \textit{Y}. Os conjuntos $A_{1}$, $A_{2}$, $A_{3}$ são chamados objetos, atributos e modo, respectivamente. O conjunto de todos os conceitos de um contexto triádico parcialmente ordenado formam um reticulado completo, chamado também de reticulado conceitual.

\begin{table}[h]
\centering
\caption{Contexto Triádico representado por uma tabela cruzada}
\label{triadictable}
\begin{tabular}{|c|c|c|c|c|c|c|c|c|c|}
\hline
\textit{\textbf{$K_{1}$/$K_{2}$-$K_{3}$}} & \multicolumn{3}{c|}{\textbf{\textit{$c_{1}$}}} & \multicolumn{3}{c|}{\textbf{\textit{$c_{2}$}}} & \multicolumn{3}{c|}{\textbf{\textit{$c_{3}$}}} \\ \hline
    & \textbf{\textit{$a_{1}$}}     & \textbf{\textit{$a_{2}$}}     & \textbf{\textit{$a_{3}$}}    & \textbf{\textit{$a_{1}$}}     & \textbf{\textit{$a_{2}$}}     & \textbf{\textit{$a_{3}$}}    & \textbf{\textit{$a_{1}$}}     & \textbf{\textit{$a_{2}$}}     & \textbf{\textit{$a_{3}$}}    \\ \hline
\textit{\textbf{$o_{1}$}}  & $\times$      & $\times$      &       &        & $\times$      &       & $\times$      &        &       \\ \hline
\textit{\textbf{$o_{2}$}}   &        &        & $\times$     &        &        & $\times$     & $\times$      &        & $\times$     \\ \hline
\textit{\textbf{$o_{3}$}}   & $\times$      &        &       & $\times$      &        &       &        & $\times$      &       \\ \hline
\end{tabular}
\end{table}


 
A partir de contextos triádicos, é possível obter regras do tipo  \textit{Biedermann Conditional Attribute Association Rule} (BCAAR)
e \textit{Biedermann Attributional Condition Association Rule} (BACAR):
\begin{enumerate}
    \item BCAAR: (A1 → A2)B(sup, conf) $\rightarrow$ considerando A1 e A2 subconjuntos de atributos, (A1, A2 $\subseteq$ K2) e B uma condicional, (B $\subseteq$ K3). Ou seja, se o atributo A1 ocorre com a condição B, A2 também irá ocorrer, com um suporte (sup) e confiança (conf).
    \item BACAR: (B1 → B2)A(sup, conf) $\rightarrow$  considerando B1 e B2 condicionais, (B1,B2 $\subseteq$ K3),  e A um atributo, (A $\subseteq$ K2). Ou seja, se B1 ocorre para os atributos A, então a condição B2 também ocorre com um suporte (sup) e confiança (conf).
\end{enumerate}

Note, novamente, que o suporte (sup) é a porcentagem de vezes em que a regra aconteceu dentre todas as instâncias analisadas, enquanto confiança (conf) é a porcentagem de vezes em que o consequente da regra foi verdadeiro quando a premissa ocorreu. 









 \subsection{Análise de Dados Longitudinais}

 Bases de dados longitudinais são conjuntos de registros onde o tempo é um parâmetro de análise. Nessas bases de dados, verifica-se a variação de atributos e características, observando suas atualizações durante períodos de tempo. Isso abre possibilidades de explorar relações causa-efeito, o que torna esse objeto de estudo relevante.

%\textcolor{red} {refazer: ESTÁ muito resumido! Falar mais de bases longitudinais. Dar exemplos.  Coloque o novo texto em AZUL!!}


\begin{table}[h!]
\centering
\begin{tabular}{ p{1cm}p{1cm}p{1cm}p{1cm}p{1cm}p{1cm}p{1cm}  }
 \hline
 \footnotesize{Objeto}& 
 \footnotesize{P1-t1}&
 \footnotesize{P2-t1}&
 \footnotesize{P3-t1}&
  \footnotesize{P1-t2}&
  \footnotesize{P2-t2}&
  \footnotesize{P3-t2}
 \\
\hline
\footnotesize{1}& 
\footnotesize{6}&
\footnotesize{6}&
\footnotesize{6}&
\footnotesize{3}&
\footnotesize{4}&
\footnotesize{3}
\\
\hline
\end{tabular}
\caption{Exemplo de base longitudinal.}
\label{table:Table II}
\end{table}

A tabela acima demonstra um fragmento da base de dados escolhida para este trabalho. O fragmento em questão traz a variação dos atributos (P1, P2 e P3) de um funcionário (Objeto), ao longo de duas ondas(t1 e t2), tornando possível verificar como os dados se comportam entre esses períodos de tempo. Por exemplo, o atributo P1 passou do grau de satisfação de 6 para 3, entre a primeira e a segunda onda. %Considerando uma relação crescente de 1 até 7 para esses atributos, podemos concluir que o atributo P1 caiu entre t1 e t2.}

\section{Trabalhos Relacionados}
%\textcolor {red} { apresentar mais uns 3 trabalhos de cada tópico: FCA/TCA com longitudinal e ABW!!! Coloque em azul o novo texto!!! Isto vai aumentar mais umas 6 referências! Procure as mais recentes!!!}


Atualmente, não foi possível encontrar grande quantidade de trabalhos em que são aplicadas regras triádicas. Os artigos \cite{Lana2022} e \cite{Noronha2022}, provavelmente são os primeiros trabalhos a explorar a análise triádica para descrever estudos longitudinais na área da saúde. O primeiro artigo trata da análise da efetividade de métodos de prevensão frente a infecção das COVID-19, e o segundo trabalho trata da descoberta de padrões acerca do envelhecimento humano observando a evolução temporal clínica e de condições ambientais dos indivíduos.

Como trabalhos relacionados ao tópico, tem-se  o estudo de \cite{article} em que uma série de experimentos foram realizados comparando resultados e desempenho de algoritmos de análise de contextos triádicos a fim de entendê-los, propor modificações, e demonstrar casos de uso onde resultados foram ou não satisfatórios. Além deste, pode-se citar o trabalho de \cite{missaoui2017lattice} em que a ferramenta Lattice Miner é proposta para a produção de regras de associação triádicas, incluindo implicações.  

No campo dos ABW, tem-se diversos estudos imersos em diferentes metodologias. A grande maioria  se beneficia dos estudos longitudinais, utilizando a variação das métricas no decorrer do tempo como um instrumento de análise. Os trabalhos \cite{doi:10.1080/00140139.2017.1398844} e \cite{Work}, ambos de caráter longitudinal, conduzem a um bom conhecimento introdutório do tema e detalham casos de estudo da implantação desse tipo de método nos ambientes de trabalho, bem como adoção de práticas mais flexíveis nos mesmos. Nestes trabalhos, funcionários são submetidos a questionários, onde as respostas são usadas para a definição de métricas de comparação. Os resultados obtidos fornecem valores e referências para comparação e composição do contexto triádico deste artigo apresentado.

Além disso, este trabalho se diferencia de estudos tais como \cite{Arundell} e \cite{Mixed}, que utilizaram \textit{Linear Mixed Models(LMM)}, e \cite{Annu} e \cite{cubic}, que utilizaram modelos de Regressão Linear. Embora todos eles sejam estudos conduzidos sob uma base longitudinal e possuam formas de coleta de dados semelhantes (questionários), o estudo proposto nesse artigo se difere quanto a utilização da ATC (Análise Triádica de Conceitos) como base da metodologia.

\section{Metodologia}

A metodologia proposta neste trabalho tem como objetivo descrever relações temporais entre as variáveis do questionário de satisfação dos funcionarios em relação a sua satisfação no tipo de ambiente de trabalho, entre as ondas, utilizando regras triádicas de associação com porcentagem mínima de confiança.

\subsection{Materiais}

A base de dados utilizada para o estudo é proposta em \cite{HALLDORSSON2022107920}, que contém dados de 100 funcionários que trabalham para uma empresa estatal que estava implantando ABW. Essa base oferece 43 atributos de questionários, incluindo as respostas para cada tópico de interesse. Na base de dados, tem-se 3 ondas: a primeira onda para 2 meses antes da implantação de ABW, a segunda para 4 meses após a implantação e a última para 9 meses após a adoção desta abordagem. Contudo, observando a base de dados, percebe-se que nem todos os funcionários responderam a todo o questionário em sua totalidade, apresentando uma taxa de resposta de 87\%, 75\%, e 69\%, respectivamente entre as ondas. Na Table \ref{table:Table I} são mostradas os atributos que compõem a base de dados, segmentados quanto a área relacionada a cada item do questionário. 

\begin{table}[h!]
\centering
\begin{tabular}{ |p{3cm}|p{9cm}|  }
 \hline
 \multicolumn{2}{|c|}{Atributos} \\
 \hline
 \footnotesize{Produtividade}& \footnotesize{Três itens relacionados ao impacto na produtividade.}\\
 \hline
\footnotesize{Privacidade}& \footnotesize{Seis itens relacionados ao impacto na privacidade, sendo ela focada na tarefa ou na comunicação.}\\
 \hline
\footnotesize{Propriedade 

Psicológica}& \footnotesize{Quatro itens relacionados ao sentimento de posse no ambiente de trabalho.}\\
 \hline
 \footnotesize{Satisfação}& \footnotesize{Dois itens relacionados a satisfação do trabalho exercido.}\\
 \hline
 \footnotesize{ABW}& \footnotesize{Um único item relacionado a gostar ou não de ABW.}\\
 \hline
 \footnotesize{Carga de Trabalho}& \footnotesize{Quatro itens relacionados a carga de trabalho.}\\
 \hline
  \footnotesize{Estresse}& \footnotesize{Quatro itens relacionados ao estresse provocado pelo trabalho.}\\
 \hline
 \footnotesize{Troca de ambiente}& \footnotesize{Um único item que questiona a frequência da mudança de ambiente de trabalho.}\\
 \hline
 \footnotesize{Efeitos na performance}& \footnotesize{Estimativa pessoal do efeito do ambiente de trabalho na performance.}\\
 \hline
 \footnotesize{Genero}& \footnotesize{O gênero do funcionário.}\\
 \hline
 \multicolumn{2}{|c|}{Condições} \\
 \hline
 \footnotesize{t1}& \footnotesize{Medida antes da implantação de ABW - 2 meses antes.}\\
 \hline
  \footnotesize{t2}& \footnotesize{Medida depois da implantação de ABW - 4 meses depois.}\\
 \hline
  \footnotesize{t3}& \footnotesize{Medida depois da implantação de ABW - 9 meses depois.}\\
 \hline
\end{tabular}
\caption{Variaveis disponíveis na base de dados.}
\label{table:Table I}
\end{table}

\subsection{Métodos}
\subsubsection{Pré-processamento}
Antes de efetivamente iniciar a extração das regras triádicas na base dados, foi aplicado um pré-processamentos na base de dados. Primeiramente, e como discutido anteriormente, a base de dados não possui 100\% de adesão nas respostas durante as 3 ondas. Sendo assim, funcionários com muitos dados faltando, em mais de uma onda, foram desconsiderados, o que reduziu o tamanho da base de 100 para 83 amostras. 

Além disso, como a base ainda possui valores vazios em alguns dos atributos, a estratégia de imputar esses valores foi utilizada, preenchendo esses campos com o valor médio das respostas para esse mesmo atributo, numa mesma onda. Por exemplo, se algum funcionário não respondeu alguma questão na primeira onda (t1), preencheu-se este dado com a média deste atributo relativo a t1. Após a aplicação dessas etapas de pré-processamento foi possível obter uma base mais consistente e robusta para a análise.

Devido a grande quantidade de atributos na base, sendo eles dos mais variados tópicos de relevância, optou-se por segmentar a base de dados limitada às relações entre os tópicos selecionados. Sendo assim, foi construída uma base que relaciona "Produtividade", "Satisfação do Trabalho" e "Carga de Trabalho", o que permitiu analisar o comportamento desses temas em conjunto. Essa segmentação resultou em uma base com 83 amostras e 7 atributos.

\subsubsection{Discretização}

Após a etapa de pré-processamento, foram definidos os valores normais de referência para determinar quando um valor total do questionário seria uma satisfação negativa ou positiva. Para isso, o valor médio da escala foi utilizado, ou seja, numa escala de 1 até 7 nos valores de certo atributo, o limiar definido para positividade é todo e qualquer valor acima de 4. A Tabela \ref{table:Table II} mostra o contexto triádico, representando com um X as satisfações positivas, com o threshold seguindo a definição discutida acima.

\begin{table}[h!]
\centering
\begin{tabular}{ p{1cm}p{1cm}p{1cm}p{1cm}p{1cm}p{1cm}p{1cm}  }
 \hline
 \footnotesize{ID}& 
 \footnotesize{P1-t1}&
 \footnotesize{P3-t1}&
 \footnotesize{P1-t2}&
  \footnotesize{P3-t2}&
  \footnotesize{P1-t3}&
  \footnotesize{P3-t4}
 \\
\hline
\footnotesize{1}& 
\footnotesize{X}&
\footnotesize{X}&
\footnotesize{ }&
\footnotesize{ }&
\footnotesize{X}&
\footnotesize{ }
\\
\footnotesize{2}& 
\footnotesize{X}&
\footnotesize{X}&
\footnotesize{X}&
\footnotesize{X}&
\footnotesize{X}&
\footnotesize{X}
\\
\hline
\end{tabular}
\caption{Exemplo de tabela após os tratamentos.}
\label{table:Table II}
\end{table}

Após a obtenção do contexto triádico, gerou-se um formato especifico de entrada do software Lattice Miner para a extração das regras triádicas. Sendo assim, criou-se um arquivo JSON, contendo as relações entre objetos, atributos e condições no formato especifico descrito na documentação da ferramenta especificando os valores mínimos de suporte e confiança para a geraração das regras BCAAR e BACAR. Foi possivel com isso a descrição dos seguintes casos, bem como das  regras. Considerando  \textit{A} e \textit{B} subconjuntos de atributos da base, e as ondas do estudo longitudinal \textit{a - antes}, \textit{b - durante}, e \textit{c - depois} :\newline



\textbf{Caso 1: Relação entre os atributos antes e durante a implantação de ABW} 
\begin{enumerate}
\item ( A $\rightarrow$ B ) ab
\item ( a $\rightarrow$ b ) A
\item ( b $\rightarrow$ a ) A
\end{enumerate}
\textbf{Caso 2: Relação entre os atributos durante e após a implantação de ABW} 
\begin{enumerate}
\item ( A $\rightarrow$ B ) bc
\item ( b $\rightarrow$ c ) A
\item ( c $\rightarrow$ b ) A
\end{enumerate}
\textbf{Caso 3: Relação entre os atributos antes e após a implantação de ABW} 
\begin{enumerate}
\item ( A $\rightarrow$ B ) ac
\item ( a $\rightarrow$ c ) A
\item ( c $\rightarrow$ a ) A
\end{enumerate}

Com essas regras, se torna possível descrever e analisar os casos do estudo longitudinal. Considerando, por exemplo, a regra (1) do caso 1, tem-se: "Os funcionários que avaliaram positivamente o atributo A na primeira onda, também o avaliaram positivamente na segunda onda.". Esta regra permite avaliar o impacto mais inicial da implantação de ABW, podendo servir posteriormente como forma de verificar a adaptação dos funcionários ao novo conceito a longo prazo. Outro exemplo então seria a regra (1) do caso 3, onde entendemos: "Os funcionários que avaliaram positivamente A também o fizeram para B, antes e após a implantação de ABW.".

\section{Experimentação e Análise dos Resultados}

Após a aplicação do contexto no LatticeMiner, foi possível extrair regras BACARs e BCAARs para os casos descritos na seção anterior, possibilitando analisar as principais regras geradas.

Caso I (antes e durante):
%\subsection{Caso I (antes e durante)}
\begin{enumerate}
   \item BACARs
   \begin{enumerate}
     \footnotesize \item ( t1 \textrightarrow{} t2 ) Produtividade2 [support = 96,4\% confidence = 98,8\%]
     \item ( t1 \textrightarrow{} t2 ) Produtividade1 [support = 92,8\% confidence = 95,1\%]
     \item ( t1 \textrightarrow{} t2 ) Satisfação2 [support = 68,7\% confidence = 78,1\%]
   \end{enumerate}
   \item BCAARs
   \begin{enumerate}
       \footnotesize \item ( Produtividade2 \textrightarrow{} Satisfação1 ) t1 [support = 89,2\% confidence = 90,2\%]
       \item ( Produtividade2 \textrightarrow{} Satisfação1 ) t2 [support = 85,5\% confidence = 87,7\%]
       \item ( Produtividade1, Produtividade3 \textrightarrow{} Satisfação1 ) t1 [support = 86,7\% confidence = 90,0\%]
       \item ( Produtividade1, Produtividade3 \textrightarrow{} Satisfação1 ) t2 [support = 78,3\% confidence = 90,3\%]
       \item ( Satisfação2 \textrightarrow{} Satisfação1 ) t1 [support = 85,5\% confidence = 97,3\%]
       \item ( Satisfação2 \textrightarrow{} Satisfação1 ) t2 [support = 67,5\% confidence = 77,8\%]
   \end{enumerate}
 \end{enumerate}

Caso 2 (durante e após):
%\subsection{Caso 2 (durante e após)}
\begin{enumerate}
   \item BACARs
   \begin{enumerate}
     \footnotesize \item ( t2 \textrightarrow{} t3 ) Produtividade1 [support = 92,8\% confidence = 95,1\%]
     \item ( t2 \textrightarrow{} t3 ) Produtividade1 [support = 92,8\% confidence = 97,5\%]
   \end{enumerate}
   \item BCAARs
   \begin{enumerate}
       \footnotesize \item ( Produtividade1 \textrightarrow{} Satisfação2 ) t2 [support = 72,3\% confidence = 75,9\%]
       \item ( Satisfação2\textrightarrow{} Produtividade1 ) t3 [support = 62,7\% confidence = 98,1\%]
       \item ( Satisfação1 \textrightarrow{} Satisfação2 ) t2 [support = 67,5\% confidence = 77,8\%]
       \item ( Satisfação1 \textrightarrow{} Satisfação2 ) t3 [support = 60,2\% confidence = 96,2\%]
   \end{enumerate}
 \end{enumerate}

 Caso 3 (antes e após):
%\subsection{Caso 3 (antes e após)}
\begin{enumerate}
   \item BACARs
   \begin{enumerate}
     \footnotesize \item ( t1 \textrightarrow{} t3 ) Produtividade1 [support = 95,2\% confidence = 97,5\%]
     \item (t3 \textrightarrow{} t1 ) Satisfação2 [support = 62,7\% confidence = 98,1\%]
   \end{enumerate}
   \item BCAARs
   \begin{enumerate}
       \footnotesize \item ( Satisfação2 \textrightarrow{} Satisfação1 ) t1 [support = 85,5\% confidence = 97,3\%]
       \item ( Satisfação1 \textrightarrow{} Satisfação2 ) t3 [support = 60,2\% confidence = 96,2\%]
   \end{enumerate}
 \end{enumerate}
 
Com essas regras, é possível descrever e concluir informações acerca do estudo longitudinal. Por exemplo, analisando as regras BACARs (c) do Caso 1 e a regra (b) do Caso 3, pode-se perceber que a satisfação com o trabalho diminui, já que os funcionários que responderam positivamente a Satisfação2 ("Estou satisfeito com meu trabalho") o fizeram com menor frequência entre as ondas (antes \textrightarrow{} durante) e (antes \textrightarrow{} após) com o suporte caindo de 68,7\% para 62,7\%.

Outro exemplo interessante seria a regra BACAR (b) do Caso 1 em conjunto com a regra BACAR (a) do Caso 3. Neste contexto, pode-se inferir que a eficiência geral caiu, como pode ser constatado analisando o suporte entre a primeira e a terceira onda, em que o suporte cai para 95,2\%. Isso indica que funcionários que se sentiam eficientes já não se consideram mais. Porém, este suporte cresce em relação ao resultado na segunda onda, de 92,8\%, mostrando que certa eficiência foi recuperada até o fim do estudo, destacando uma possível habituação dos funcionários ao novo método de trabalhar.

A regra BACAR (a) do Caso 1, em conjunto com a regra BACAR (a) do Caso 2, indicam que a eficiência caiu para colaborações, já que os funcionários que responderam positivamente a Produtividade2 ("Sinto que trabalho de forma eficiênte quando colaboro com meus colegas") o fizeram com menor frequência entre as ondas (antes \textrightarrow{} durante) e (durante \textrightarrow{} após) com o suporte caindo de 96,4\% para 92,8\%.

 
\section{Conclusões e Trabalhos Futuros}

Nesse trabalho, foi mostrado o potencial da aplicação de regras triádicas para descrição de bases de dados longitudinais em contextos sociais, onde as mudanças na forma de trabalhar, como descrito nesse artigo, resultam em implicações diversas. Foram geradas, nesse contexto, regras que possibilitam entender melhor o comportamento dos funcionários antes, durante e após a aplicação de ABW.

É importante  destacar que a Análise de Conceitos Triádicos tem se mostrado relevante na tomada de decisão e na análise geral de bases longitudinais. A aplicação da mesma em bases de maior dimensionalidade trará maior precisão e descrição dos dados, proporcionando conclusões ainda mais significativas.

Como trabalhos futuros, seria interessante realizar  estudos que envolvam intervalos de tempos que possibilitem  a existência de um número maior de ondas como forma de observar se os funcionários se acostumam e se adaptam a nova forma de trabalhar. Isto poderia  mitigar os  efeitos negativos do desempenho e satisfação causados pelo choque da mudança de trabalho.

Além disso, poder-se-ia estender a análise a várias empresas tornando possível avaliar a contribuição da cultura interna da mesma e de seus funcionários no impacto, negativo ou positivo, da aplicação de ABW visto que o caráter do tema apresenta lacunas por possibilitar  interpretação pessoal das mudanças por parte dos participantes.
	
% INCLUDE BIBLIOGRAPHY WHICH MUST FOLLOW kdmile.bst TEMPLATE
\bibliographystyle{kdmile}
\bibliography{kdmileb}
% For information on how to write bibliography entries, 
% see file kdmileb.bib

\begin{received}
\end{received}

\end{document}
