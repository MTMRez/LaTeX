\documentclass[12pt]{article}

\usepackage{sbc-template}

\usepackage{graphicx,url}
\usepackage{amsmath}
%\usepackage[brazil]{babel}   
\usepackage[utf8]{inputenc}

\usepackage[portuguese]{babel}

\sloppy

\title{Minimizar distância percorrida por entregadores de pizza}

\author{Marcos Tadeu Magalhães Rezende\inst{1}, Gabriella Mara Pereira\inst{1}, Adriano Rodrigues de Assis Vaz\inst{1}}


\address{Pontifícia Universidade Católica de Minas Gerais}

\begin{document} 

\maketitle

\section{Ideia}

\begin{itemize}
    \item Calcular a menor rota, considerando uma quantidade de entregadores disponíveis
    \item Usar a API do Google para pegar a distância entre dois endereços
    \item Input:
    \begin{enumerate}
        \item Número de pedidos
        \item Localização de cada pedido 
        \item Número de entregadores disponíveis
    \end{enumerate}
\end{itemize}

\section{Variáveis envolvidas}
i → endereço de origem (0,…,N)
\\j → endereço de destino (0,…,N)
\\Dij → distância entre o endereço de origem i e o de destino j
\\Xijk → se o pedido do endereço i já foi entregue por um entregador k
\\M → número de entregadores disponíveis
\\k → entregador em questão (1,…,M)
\\N → número total de locais dos pedidos 

\section{Variáveis de decisão}
Xijk → 1 se o entregador k está indo do endereço i ao j, 0 caso contrário

\section{Objetivo}
O objetivo a ser alcançado é de minimizar a distância a ser percorrida para que todas as entregas da pizzaria sejam realizadas.

Função objetivo:
$$
Minimizar D = \sum_{i=0}^N \sum_{j=0}^N \sum_{k=1}^M Dij x Xijk
$$

De exemplo, vamos imaginar a seguinte condição:
3 entregas
2 entregadores

\begin{align*}
Min D = D01X011 + D02X021 + D03X031 
\\ + D12X121 + D13X131 + D21X211 
\\ + D23X231 + D31X311 + D32X321
\\ + D01X012 + D02X022 + D03X032 
\\ + D12X112 + D13X132  + D21X212 
\\ + D23X232 + D31X312 + D32X322
\end{align*}

Entregador 1 → melhor caminho 0 → 1 → 2

Entregador 2 → melhor caminho 0 → 3

\begin{align*}
Min D = D01x1 + 0 + 0 + D12x1 + 0 + 0 + 0 + 0 + 0
\\ + 0 + 0 + D03x1 + 0 + 0 + 0 + 0 + 0 + 0
\end{align*}

$$
Min D = D01 + D12 + D03
$$

\section{Restrições}

Mas para isso, devemos levar em consideração algumas restrições. Primeiro que a pizzaria deve entregar todos os pedidos e cada um deve ser entregue apenas uma vez por um único entregador. Como o entregador só pode ir pra um endereço j se existe alguma rota do endereço atual i, precisamos criar uma restrição para evitar que o nosso modelo crie sub rotas. Por exemplo, o modelo poderia considerar uma rota de entrega ótima como D12 → D25, D34 → D41, sendo que é impossível ele sair do endereço 5 e ir direto para o 3 sem percorrer nenhuma distância. 

Vamos também desconsiderar as diferenças de cada moto, assim como capacidade, combustível, entre outros.

Por isso sujeitamos o nosso problema a:
\begin{equation}
\sum_{i=0}^N \sum_{k=1}^M Xijk = 1 | \forall j = 1,...,N
\end{equation}
\begin{equation}
\sum_{j=0}^N \sum_{k=1}^M Xijk = 1 | \forall i = 1,...,N
\end{equation}
\begin{equation}
\sum_{j=0}^N X0jk = 1 | \forall k = 1,...,M
\end{equation}
\begin{equation}
\sum_{i=0}^N Xi0k = 1 | \forall k = 1,...,M
\end{equation}
\begin{equation}
Xijk \in {0,1} | \forall i,j = 0,...,N, k = 1,...,M
\end{equation}
\begin{equation}
Dij \geq 0 | \forall i,j = 0,...,N
\end{equation}
\begin{equation}
i \in {0,...,N}
\end{equation}
\begin{equation}
j \in {0,...,N}
\end{equation}
\begin{equation}
i, j, k, N, M, D > 0
\end{equation}
As equações 1 e 2 garantem que cada endereço seja percorrido somente uma vez por um único entregador. Já as equações 3 e 4 garantem que cada entregador só pode sair da pizzaria uma vez. A equação 6 garante que a distância entre dois endereços não pode ser negativa, já a 7 e 8 garantem que o endereço a ser visitado é um dos locais de entrega da pizzaria.


\end{document}
