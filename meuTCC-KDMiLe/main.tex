

% by Mirella M. Moro; version: January/18/2012 @ 04:16pm
% -- 01/18/2012: more discussion on SBBD + kdmile; overall revision
% -- 09/03/2010: bib file with names for proceedings and journals; cls with shrinked {received}
% -- 08/27/2010: appendix, table example, more explanation within comments, editors' data

\documentclass[kdmile,a4paper]{kdmile} % NOTE: kdmile is published on A4 paper
\usepackage{graphicx,url}  % for using figures and url format
\usepackage[T1]{fontenc}   % avoids warnings such as "LaTeX Font Warning: Font shape 'OMS/cmtt/m/n' undefined"
\usepackage{xcolor}

%\usepackage{cite} % NOTE: do **not** include this package because it conflicts with kdmile.bst

% Standard definitions
\newtheorem{theorem}{Theorem}[section]
\newtheorem{conjecture}[theorem]{Conjecture}
\newtheorem{corollary}[theorem]{Corollary}
\newtheorem{proposition}[theorem]{Proposition}
\newtheorem{lemma}[theorem]{Lemma}
\newdef{definition}[theorem]{Definition}
\newdef{remark}[theorem]{Remark}

% New environment definition
\newenvironment{latexcode}
{\ttfamily\vspace{0.1in}\setlength{\parindent}{18pt}}
{\vspace{0.1in}}

% ALL FIELDS UNTIL BEGIN{document} ARE MANDATORY


% Includes headers with simplified name of the authors and article title
\markboth{M. T. M. Rezende and L. E. Zárate}{KDMiLe - Symposium on Knowledge Discovery, Mining and Learning - Algorithms Track}
%  -> \markboth{}{}
%         takes 2 arguments
%         ex: \markboth{A.Plastino}{Any article title}


% Title of the article
\title{Triadic analysis to evaluate the impact of investments in research by companies in the technology field\\} 


% List of authors
%IF THERE ARE TWO or more institutions, please use:
%\author{Name of Author1\inst{1}, Name of Author2\inst{2}, Name of Author3\inst{2}}
\author{Marcos T. M. Rezende\inst{1}, Mark A.J. Song \inst{1}, Luis E. Zárate\inst{1}}


%Affiliation and email
\institute{Pontifícia Universidade Católica de Minas Gerais, Departamento de Ciência da Computação, Brazil \\ \email{mtmrezende@sga.pucminas.br, song@pucminas.br, zarate@pucminas.br}
% IF THERE IS ANOTHER INSTITUTION:
% \and Universidade de S\~ao Paulo, Brazil  \\ \email{andre@icmc.usp.br}
% \and Universidade Federal Fluminense, Brazil  \\ \email{plastino@ic.uff.br}
}


% Article abstract - it should be from 100 to 300 words
\begin{abstract}
We know that several factors favor a company to remain active in the long term. The impact of these factors can be monitored over time, and one way to do this is through longitudinal studies. From this study, and from the database that it composes, the company can be observed through consecutive periods of time called waves. The application of triadic rules, based on the Formal Concept Analysis, is an approach to describe the different strategies of companies to remain active. As main results, we emphasize investments in research and development, and hiring of researchers among employees, when implemented together, contributed to increasing the chances of companies to remain active over the years of the study, reflecting their strategy in issuing patents and billing with exports.
\end{abstract}

%% ACM Computing Classification System categories
%% The code below is generated by the tool at http://dl.acm.org/ccs.cfm.
%% Please copy and paste the code instead of the example below.
%%
\begin{CCSXML}
	<ccs2012>
	<concept>
	<concept_id>10010147.10010257.10010321</concept_id>
	<concept_desc>Computing methodologies~Machine learning algorithms</concept_desc>
	<concept_significance>500</concept_significance>
	</concept>
	</ccs2012>
\end{CCSXML}

\ccsdesc[500]{Computing methodologies~Machine learning algorithms}


% Article keywords
\keywords{Activity in long term, Big Data, Companies, Data Mining, Formal concept analysis, Lattice Miner, Longitudinal Database, Machine Learning, Triadic concept analysis, Triadic rules}
%  -> \keywords{} (in alphabetical order \keywords{document processing, sequences,
%                      string searching, subsequences, substrings})



% THE ARTICLE BEGINS
\begin{document}

% This is optional:
\begin{bottomstuff}
% similar to \thanks
% for authors' addresses; research/grant statements
\end{bottomstuff}

\maketitle

\section{Introdução}

Sabemos que vários fatores favorecem uma empresa a se manter ativa a longo prazo, como sua organização, seu faturamento, o investimento em inovação, expansão do negócio e influência no mercado. Uma análise da evolução desses fatores é fundamental para entender o perfil dessas empresas e avaliar para quais setores deve-se redirecionar os investimentos na expectativa de manter a empresa ativa pelos próximos anos. Para analisar a relação desses fatores, os estudos mais adequados correspondem aos denominados estudos longitudinais.

Estudos longitudinais, amplamente utilizados na área social, correspondem aos estudos onde as observações de uma mesma amostra (empresas, neste trabalho) são registradas em períodos de tempo distintos, e consecutivos, chamados ondas. Os estudos longitudinais na área empresarial permitiriam acompanhar a evolução das empresas envolvidas no estudo. Por exemplo, uma base de dados longitudinal pode registrar informações acerca das condições de investimento em tecnologia na empresa, a quantidade de profissionais com nível de doutorado em relação a outros profissionais, o faturamento anual da empresa, entre outros fatores, para diferentes momentos temporais (ondas, neste trabalho). Essas bases, seriam uma fonte valiosa para descrever, analisar e validar procedimentos, e sobretudo para extrair conhecimentos e informação útil, e às vezes não óbvias, para apoio à tomada de decisão.

Para analisar as bases de dados de estudos longitudinais, propomos neste trabalho, utilizar a teoria da Análise Formal de Conceitos (FCA) \cite{ganter2012formal}, um braço da matemática aplicada, fundamentada na teoria dos reticulados conceituais, cujo principal objetivo é representar e extrair conhecimento a partir de um conjunto de dados envolvendo objetos (empresas), atributos (características das empresas como faturamento, investimento, etc) e suas relações de incidência. Essa tupla de elementos pode ser representada por meio de um contexto formal diádico, a partir do qual é possível extrair regras de associação  entre os atributos. A Análise de Conceitos Triádicos (TCA), proposta por \cite{lehmann:95}, é uma extensão da teoria da FCA, a qual utiliza o contexto formal triádico, que introduz um terceiro elemento, uma condição que determina a relação de incidência entre objetos e atributos. No contexto organizacional, objetos podem corresponder a empresas, atributos as características (fatores) dessas empresas, e condições às diferentes ondas do estudo longitudinal. A partir do contexto triádico é possível extrair regras de associação triádicas que podem ser utilizadas para observar as relações entre esses atributos numa dada onda ou entre ondas. 

Embora nos últimos anos, FCA tenha recebido grande atenção na área de Mineração de Dados, a aplicação da teoria triádica tem sido ainda muito restrita, e totalmente ausente na descrição analítica de estudos longitudinais. Desta forma, descrever as características das empresas submetidos a um estudo longitudinal para análise do impacto do investimento em pesquisa, pode ser relevante para entender os efeitos das ações e tomadas de decisão das empresas.

Em 2019, foi publicado um estudo em \cite{KIM2019103967} que registrou dados acerca de 588 pequenas e médias empresas instaladas em 3 polos tecnológicos-regionais de inovação na Coreia do Sul. Esse estudo considera o tamanho da empresa, o setor industrial no qual ela atua, a inovação tecnológica, e o tempo de sobrevivência da mesma. A análise da relação desses dados pode ser considerada relevante para estudos da efetividade da área de inovação nas pequenas e médias empresas. Os dados foram coletados anualmente entre os anos de 2008 a 2014 e agrupados numa base de dados longitudinal. O estudo foi parte de uma iniciativa do governo coreano de criação e apoio de ecossistemas em polos tecnológicos-regionais. Esse estudo ser´considerado como estudo de caso para mostar a aplicabilidade da análise triádica.

A partir disso, o objetivo deste trabalho é aplicar a teoria da análise formal de conceitos, especificamente a análise triádica sobre os dados fornecidos pelo estudo longitudinal \cite{KIM2019103967} na expectativa de obter potenciais relações entre o fator investimento em pesquisa e desenvolvimento tecnológico das empresas, e das suas chances de sobrevivência a longo prazo.

O presente artigo está dividido em cinco seções. Na seção 2, uma breve revisão de trabalhos relacionados é fornecida. Na seção 3, a fundamentação teórica que sustenta a nossa proposta é apresentada. Na seção 4, a metodologia experimental é descrita e na seção 5, os resultados experimentais são discutidos. Finalmente as conclusões e trabalhos futuros são apontados.

%A seguir, serão apresentados as referências teóricas na qual este trabalho é fundamentado, quais ferramentas serão utilizadas, como serão utilizadas e quais resultados pretende-se obter, além da apresentação dos resultados e como eles podem ser aproveitados em trabalhos futuros.

\section{Trabalhos relacionados} \label{sec:firstpage}

%A análise formal de conceitos triádicos é mais comumente aplicada na área da saúde, onde estudos longitudinais, formato ideal para trabalhar nesse campo de estudos, são mais promissores. Em \cite{mendes:21}, a AFC triádica foi utilizada para descrever as relações entre os sintomas apresentados ao longo de um tratamento da doença de Parkinson. Em \cite{noronha:21}, a análise triádica foi utilizada num estudo para identificar a evolução de condições que favorecem os perfis de longevidade baseado num estudo britânico sobre envelhecimento. Há também o trabalho de \cite{ferreira:21}, que analisou os efeitos do uso de olanzapina em um grupo de pacientes em tratamento quimioterápico para câncer de mama.

%No entanto, há aplicações da análise formal de conceitos em outras áreas, como a própria computação.Em \cite{dias:11}, é proposto o arcabouço EF-Concept Analysis, uma ferramenta para auxiliar no desenvolvimento e avaliação de algoritmos para análise formal de conceitos.

%No trabalho proposto em \cite{dias:21}, apresenta-se a metodologia FCANN, baseada na análise formal de conceitos, usada para extração de conhecimento de processos treinados por redes neurais artificiais.

Desde que foi formalizada, FCA tem sido constantemente desenvolvida e empregada em diversas aplicações, como Information Retrieval \cite{Kumar2012}, Security Analysis \cite{Poelmans2013}, Text Mining \cite{DeMaio2014}, Topic Detection \cite{Cigarran2016}, Social Network Analysis \cite{MissaouiSergei2017}, Search Engine \cite{Negm2017}, entre outras \cite{Singh2011}. Em geral, FCA tem sido utilizada em análise de dados e representação de conhecimento, onde associações e dependências são identificados a partir de uma relação binária de incidência entre objetos e atributos \cite{c01,c10}. Diversos artigos \cite{Disc1998,u07,u59,u50} também discutiram o uso da extração de regras de associação por meio da FCA.

Uma vez que os conjuntos de dados são frequentemente expressos por relações ternárias e mais geralmente $n$-árias, pode-se observar um recente interesse em propor novas soluções para análise e exploração desses dados multidimensionais, especialmente em contextos triádicos \cite{Bazin2020,Felde2021}. Em \cite{trias01}, os autores apresentaram várias estratégias para descoberta de padrões ótimos considerando conjuntos de dados triádicos.

Em \cite{kaio:21} os autores avaliaram o algoritmo TRIAS \cite{Trias}, que executa análise triádica a partir de projeções em contextos triádicos de alta dimensionalidade, principalmente no uso de diagramas binários de decisão (BDD), uma representação canônica mais compacta para fórmulas booleanas.

Em estudos voltados para empresas, as técnicas tradicionais de machine learning são mais comuns. Por exemplo, o trabalho de \cite{lima:24} aplicou regressão linear e análise discriminante num estudo com cerca de 2 mil empresas. O objetivo dos autores foi avaliar as relações entre fatores ESG (Environmental, Social and Governance) e a saúde financeira das empresas.

Em \cite{falqueto:22}, foi proposta uma abordagem baseada em clusterização para segmentar clientes de e-commerce e avaliar o potencial financeiro desses grupos. O trabalho \cite{campos:21} apresenta algumas aplicações de algoritmos clássicos do machine learning, como árvore de decisão, regressão linear e SVM na industria farmacêutica. Já \cite{vidal:20} faz uma comparação de eficiência entre métodos de tomada de decisão multi-critério e machine learning em diferentes casos para gestão de estoques.

Pelo exposto anteriormente, e considerando o maior de nosso entendimento, não foi possível encontrar contribuições da Análise Formal de Conceitos, especialmente da Análise Triádica voltada para estudos longitudinais em ambientes organizacionais. Desta forma pretendemos contribuir com a análise triádica para sustentar sistemas decisórios na gestão das empresas.

\section{Fundamentação teórica}

Estudos longitudinais consistem em análise de observações de uma mesma amostra de objetos realizadas em períodos de tempo consecutivos denominados ondas. Estas observações são armazenadas em bases de dados chamadas longitudinal.

A análise deste tipo de base de dados gera informação sobre a qual pode-se concluir regras de associação com capacidade preditiva ou apenas tarefas descritivas conforme a quantidade de registros presentes na base. Essas regras de associação são obtidas a partir da análise formal de conceitos (AFC) \cite{wille:95}, um ramo da matemática aplicada baseado na hierarquização do conceito formal, a partir de um conjunto de objetos e suas propriedades, visando representar e extrair conhecimento a partir da mesma. A sua extensão, a análise de conceitos triádicos (ACT) \cite{lehmann:95}, procura extrair regras de associação/implicação com condições baseado nas variações das propriedades dos objetos ao longo das ondas.

As principais definições da FCA são o contexto formal, na qual se identifica uma relação de incidência $I$ entre um conjunto de objetos $G$ e seu conjunto de atributos $M$, definida por $(I\subseteq{G \times M})$, e o conceito formal $(A,B)$, na qual extrai-se dois sub-conjuntos $A$ e $B$, denominados extensão e intenção, derivados respectivamente de $G$ e $M$, na qual cada elemento de um sub-conjunto possui todos os elementos do outro, descrito por $(A \subseteq{G} \ e \ B \subseteq{M} \iff A' = B \ e \ B' = A)$.

Já a teoria da análise conceitual triádica, proposta por \cite{lehmann:95}, é uma abordagem tridimensional da análise formal de conceitos, cujo contexto formal estabelece uma relação ternária $I$ entre três conjuntos $K1$, $K2$ e $K3$, descrito pela quádrupla $(K1, K2, K3, I)$. Em outras palavras, $I \subseteq K1 \times K2 \times K3$, onde os elementos de $K1$, $K2$, e $K3$ são chamados objetos, atributos e condições, respectivamente. Esta relação pode ser representada por $(g,m,b) \in I$, e pode ser lido como: “o objeto $g$ possui o atributo $m$ sob a condição $b$”. Um exemplo do contexto triádico é mostrado na Tabela \ref{Tabela:1}, onde podemos considerar as empresas como os objetos, cada onda como uma condição, e os tipos de investimentos como os atributos. Assim, é possível registrar as empresas, suas ações de investimento, e seu faturamento para cada onda, ou entre ondas para análise da efetividade do investimento.


\begin{table}[h!] 
\scriptsize
\begin{center}
\caption{Triadic context example} \label{Tabela:1}
\begin{tabular}{c|cc|cc|cc} \hline
&\multicolumn{2}{c|}{\textbf{Wave1}}&\multicolumn{2}{c|}{\textbf{Wave2}}&\multicolumn{2}{c}{\textbf{Wave3}}
\\ Company &Factor1&Factor2&Factor1&Factor2&Factor1&Factor2\\
\hline
1   & X &  
    &   & X
    &   & X \\
2   & X &  
    & X & X
    &   &   \\
3   &   & X 
    & X & X
    & X &   \\
    \hline
\end{tabular}
\end{center}
\end{table}
%\vspace{-4mm}

Em ACT, um contexto formal triádico pode ser transformado num contexto formal diádico com restrições, onde uma incidência entre objeto (Empresa), atributo (Aporte Financeiro, Investimento Pesquisa, etc), numa condição específica (onda no estudo longitudinal) não pode ser excludente para outras características para a mesma onda. A Tabela \ref{Tabela:1b} apresenta a transformação do contexto formal triádico, mostrado na Tabela \ref{Tabela:1}, para o contexto formal diádico.

\begin{table}[h!] 
\scriptsize
\begin{center}
\caption{Example of transformation from triadic formal context to dyadic formal context} \label{Tabela:1b}
\begin{tabular}{c|cc|cc|cc} 
\hline
%&\multicolumn{2}{c}{\textbf{Wave1}}&\multicolumn{2}{c}{\textbf{Wave2}}&\multicolumn{2}{c}{\textbf{Wave3}}
%\\
Company &Factor1-W1&Factor2-W1&Factor1-W2&Factor2-W2&Factor1-W3&Factor2-W3\\
\hline
1   & X &  
    &   & X
    &   & X \\
2   & X &  
    & X & X
    &   &   \\
3   &   & X 
    & X & X
    & X &   \\
    \hline
\end{tabular}
\end{center}
\end{table}

Da quádrupla $(K1, K2, K3, I)$ , pode-se extrair dois tipos de regra de associação triádicas propostas por \cite{biedermann:97}: Biedermann Conditional Attribute Association Rule (BCAAR) e Biedermann Attributional Condition Association Rule (BACAR).

A regra BCAAR, representada por: ($R$→$S$)$C$ (sup, conf), corresponde ao sub-conjunto de objetos possuindo todos os atributos em $R$, possuem também todos os atributos em $S$ sob a condição (onda) $C$, com um suporte (sup) e uma confiança (conf). Já a regra BACAR, representada por: ($P$→$Q$)$N$ (sup, conf), corresponde ao sub-conjunto de objetos, sob as condições em $P$, também estará sob as condições em $Q$ no sub-conjunto de atributos $N$, com um suporte (sup) e uma confiança (conf).

\begin{itemize}
    \item \textbf{Support}: A métrica de suporte corresponde à proporção de objetos no subconjunto $g \in G$ que satisfazem a implicação \textit{$P \rightarrow Q$}, em relação ao número total de objetos $\mid G\mid$ do contexto formal $\mathcal{K}$, onde $(.)'$ corresponde ao operador de derivação definido na Equação \ref{OpB}.
 
\begin{equation}
\label{support}
Support(P \to Q) = \frac{ \mid (P \cup \{Q\})' \mid }{\mid G \mid}
\end{equation}
\end{itemize}


\begin{itemize}
    \item \textbf{Confiance}: A métrica de confiança corresponde à proporção dos objetos $g \in G$ que contêm $P$, e também contêm $Q$, em relação ao número total de objetos $\mid G\mid$.

\begin{equation}
\label{confidence}
Confiance(P \to Q) = \frac{\mid(P \cup \{Q\})' \mid}{ \mid P' \mid}
\end{equation}
\end{itemize}



%O suporte corresponde à proporção de objetos no subconjunto que satisfaz a implicação P→Q, em relação ao número total de objetos do contexto formal. Já a confiança corresponde à razão dos objetos deste subconjunto que contêm um atributo P e que também contêm outro atributo Q, em relação ao número total de objetos.

\section{Metodologia de Experimentos}

A base de dados utilizada como estudo de caso (\cite{KIM2019103967}) coletou dados ao longo de 7 anos para 588 pequenas e médias empresas instaladas em três polos tecnológicos-regionais de inovação na Coreia do Sul. As informações coletadas das empresas para o estudo longitudinal são listadas a seguir:

\begin{itemize}
    \item Número de empregados;
    \item Faturamento total;
    \item Investimento em Pesquisa e Desenvolvimento Geral;
    \item Investimento em Pesquisa e Desenvolvimento Interno;
    \item Número de profissionais envolvidos em pesquisa;
    \item Emissão de patentes;
    \item Emissão de certificado de risco;
    \item Faturamento com exportações;
    \item Taxa de concentração do setor;
    \item Taxa de crescimento do setor.
\end{itemize}

Para analisar o efeito das ações tomadas pelas empresas, em relação a seu faturamento, foi decidido considerar ações de médio prazo (levando em consideração os três primeiros anos), e seu efeito a longo prazo (considerando os quatro anos seguintes). Desta forma foram definidos duas novas ondas para análise do estudo longitudinal.

Durante a preparação da base de dados, para cada onda, foram calculadas a porcentagem de investimentos em pesquisa e desenvolvimento interno em relação ao Faturamento total (Invest.), a porcentagem do número de pesquisadores em relação ao número total de empregados (Num-Pesq.), a quantidade total de patentes emitidas, e se houve ou não faturamento com exportações (Fat-Export.), como mostrado a seguir:

%Durante a preparação da base de dados, para cada onda, foram consideradas as seguintes combinações de variáveis:

\begin{itemize}
    \item Investimentos em Pesquisa e Desenvolvimento Interno/Total de faturamento, (Invest);
    \item Pessoal de pesquisa/Número de empregados, (Num-Pesq);
    \item Experiência com Emissão de patentes, (Patentes);
    \item Experiência de Faturamento com exportações, (Fat-Export);
\end{itemize}


Desde que é necessária a representação do novo conjunto de dados longitudinal dentro do contexto formal triádico, por meio da incidência binária \{0,1\}, foi necessário aplicar um threshold (para cada variável considerada). Para isto, foram estabelecidas thresholds baseadas na média, mediana e terceiro quartil das variáveis Invest, Num-Pesq, Patentes, e Fat-Export para as 530 empresas (90,1\% do total) que se mantiveram ativas durante todo o período da coleta de dados, ver Tabela \ref{tabela:1}. Para as condições favoráveis ao aumento da produtividade nas empresas foi atribuído valor de incidência \{1\}, caso contrário o valor \{0\}. Valores booleanos de 1 e 0 representam, respectivamente, quando o resultado alcançado pela empresa é maior ou igual ao valor estabelecido pelo threshold, ou não.

Para este estudo, o threshold que usa como limiares os valores de mediana para as variáveis apresentou os melhores resultados.

\begin{table}[ht!]
    \centering
    \begin{tabular}{||c c||} 
         \hline
         Variável & Threshold \\ [0.5ex] 
         \hline\hline
         Invest. & 2,5\% \\
         Num.Pesq. & 5,1\% \\
         Patentes & 1 \\
         Fat.Export. & 1 \\ [1ex] 
         \hline
    \end{tabular}
    \caption{Valores das medianas para as variáveis calculadas no pré-processamento em relação às empresas ativas}
    \label{tabela:1}
\end{table}

O conjunto de dados pré-processado foi usado para gerar um arquivo no formato JSON que é utilizado na aplicação da ferramenta Lattice Miner 2.0 (\cite{missaoui2017lattice}). Esta ferramenta gera conceitos formais e regras de associação triádicas a partir de relações binárias entre objetos e atributos. Este arquivo a ser utilizado pela aplicação contêm:

\begin{itemize}
    \item As listas de objetos e atributos do conjunto de dados considerado;
    \item A lista de condições, ou seja, os períodos em que os objetos foram observados, como antes (3 primeiras ondas do estudo), e após (4 últimas ondas do estudo);
    \item Mínimo de suporte desejado (opcional);
    \item As relações de incidência, ou seja, as condições que apresentam correspondência em relação ao parâmetro do threshold para cada atributo.
\end{itemize}

Após receber o arquivo JSON, a ferramenta gera uma tabela de contexto, baseado nos dados recebidos, para gerar regras de associação triádicas BCAARs ou BACARs com nível de suporte e/ou confiança.

%\subsection{Resultados esperados}

Devido à grande quantidade de regras geradas, para ser efetivo na análise triádica, é necessário definir escopos de interesse. Para este trabalho definimos os seguintes escopos: a) Relação de uma mesma variável em diferentes ondas, de forma a entender a continuidade de ações, e dos resultados alcançados antes e após ações tomadas pelas empresas; e b) Relação de diferentes variáveis numa mesma onda. De forma a analisar a relação das ações adotadas pelas empresas. A partir desses escopos foram definidos os seguintes cenários para análise.

%\begin{itemize}
 %   \item Grupo 1: Relação de um mesmo atributo em diferentes ondas;
 %   \item Grupo 2: Relação de diferentes atributos numa mesma onda.
%\end{itemize}

\begin{itemize}
    \item Cenário 1: Relação entre porcentagem de pesquisadores em relação ao número de empregados entre as ondas;
    \item Cenário 2: Relação entre múltiplas variáveis coletados na primeira onda;
    \item Cenário 3: Relação entre múltiplas variáveis coletados na segunda onda;
    \item Cenário 4: Relação entre múltiplas variáveis coletados na primeira e segunda ondas;
\end{itemize}

Para cada um destes cenários, tem-se um conjunto de regras de associação com um nível de suporte e confiança específicos.

%De forma genérica, as regras BACAR descrevem o perfil das empresas envolvidas no estudo longitudinal analisado, enquanto as regras BCAAR descrevem as regras de associação implicações, como de <Maior investimento em pesquisa> (antecedente) para <aumento da produtividade> (consequente), oriundas desse perfil.

\section{Análise dos resultados}

Para uma melhor análise dos cenários propostos, foram consideradas as regras com os maiores valores para a medida da confiança. Para o suporte consideramos que tanto valores maiores como menores podem apontar interessantes generalizações e exceções respectivamente. Estas últimas podem indicar casos de sucesso entre as empresas analisadas.

Foram produzidas 8 regras BACAR com 25.7\% a 39.3\% de suporte e 62.9\% a 86.5\% de confiança. Um resumo com as regras mais relevantes analisadas é descrito a seguir:

\begin{itemize}
    \item Regra 1: ( 1ª -> 2ª ) Fat-Export. [support = 37.1\% confidence = 86.5\%]
    %\item 2: ( b -> a ) Fat.Export. [support = 37.1\% confidence = 78.1\%]
    %\item 3: ( b -> a ) Invest. [support = 38.9\% confidence = 80.6\%]
    \item Regra 2: ( 1ª -> 2ª ) Patentes [support = 25.7\% confidence = 65.4\%]
    %\item 5: ( b -> a ) Patentes [support = 25.7\% confidence = 62.9\%]
    \item Regra 3: ( 1ª -> 2ª ) Num-Pesq. [support = 39.3\% confidence = 73.8\%]
    \item Regra 4: ( 1ª -> 2ª ) Invest. [support = 38.9\% confidence = 79.8\%]
    %\item 8: ( b -> a ) Num.Pesq. [support = 39.3\% confidence = 85.9\%]
\end{itemize}

Reiterando que 90\% das empresas abordadas neste estudo se mantiveram ativas durante todo o período analisado, observou-se, por meio da análise BACAR, que as empresas analisadas pelo estudo longitudinal, aproximadamente 1/3 delas mantêm contínuo investimento, pesquisadores no grupo de funcionários, tiveram patentes, e tiveram faturamento com exportações durante as duas ondas consideradas (antes para os 3 anos iniciais, e depois, para os 4 anos seguintes do estudo longitudinal original, correspondente a 7 anos). A princípio estas regras permitem descrever o comportamento temporal das empresas durante ambas as ondas.

%//se o perfil fosse concentrado, o suporte seria maior.

Por exemplo, a Regra 4: \textit{( 1ª -> 2ª ) Invest. [support = 38.9\% confidence = 79.8\%]} expressa que 38.9\% das empresas que investiram a partir de 2.5\% de seus faturamentos em pesquisa e desenvolvimento (threshold obtido a partir do valor da mediana da variável) nos três primeiros anos do estudo também o fizeram nos quatro anos seguintes (segunda onda). A Regra 3: \textit{( 1ª -> 2ª ) Num-Pesq. [support = 39.3\% confidence = 73.8\%]}, indica que 39.3\% das empresas que possuiam empregados pesquisadores nos 3 primeiros anos, mantiveram esse cenário nos 4 anos seguintes. Isso pode mostrar uma estratégia para a sobrevivência a longo prazo das empresas em estudo. 

Já as Regra 2: \textit{( 1ª -> 2ª ) Patentes [support = 25.7\% confidence = 65.4\%]} e Regra 1: \textit{( 1ª -> 2ª ) Fat.Export. [support = 37.1\% confidence = 86.5\%]} apresentam uma tendência a empresas que se mantiveram ativas durante todo o período do estudo terem refletido seus melhores resultados com <Faturamento com exportações> que com <Emissão de patentes>.

Também foram produzidas 56 regras BCAAR com 15\% a 42.9\% de suporte e 60.2\% a 90.7\% de confiança. As mais relevantes analisadas neste trabalho são descritas a seguir:

\begin{itemize}
    %\item 1: ( E -> P ) a [support = 26.4\% confidence = 61.5\%]
    %\item 2: ( E -> R ) a [support = 29.9\% confidence = 69.8\%]
    %\item 3: ( E -> I ) a [support = 28.9\% confidence = 67.5\%]
    %\item 4: ( E -> I ) b [support = 31.0\% confidence = 65.2\%]
    %\item 5: ( E -> P ) b [support = 26.4\% confidence = 55.6\%]
    %\item 6: ( E -> R ) b [support = 26.9\% confidence = 56.6\%]
    \item R1: ( Invest. -> Fat.Export. ) 2ª [support = 31.0\% confidence = 64.1\%]
    \item R2: ( Invest. -> Patentes ) 2ª [support = 29.1\% confidence = 60.2\%]
    \item R3: ( Invest. -> Num.Pesq. ) 2ª [support = 38.4\% confidence = 79.6\%]
    %\item 10: ( P -> E ) a [support = 26.4\% confidence = 67.1\%]
    %\item 11: ( P -> R ) a [support = 30.1\% confidence = 76.6\%]
    %\item 12: ( P -> I ) a [support = 29.6\% confidence = 75.3\%]
    %\item 13: ( P -> E ) b [support = 26.4\% confidence = 64.6\%]
    %\item 14: ( P -> I ) b [support = 29.1\% confidence = 71.2\%]
    %\item 15: ( P -> R ) b [support = 26.7\% confidence = 65.4\%]
    \item R4: ( Num.Pesq. -> Fat.Export. ) 1ª [support = 29.9\% confidence = 56.2\%]
    \item R5: ( Num.Pesq. -> Patentes ) 1ª [support = 30.1\% confidence = 56.5\%]
    \item R6: ( Num.Pesq. -> Invest. ) 1ª [support = 42.9\% confidence = 80.5\%]
    \item R7: ( Invest. -> Fat.Export. ) 1ª [support = 28.9\% confidence = 59.2\%]
    \item R8: ( Invest. -> Patentes ) 1ª [support = 29.6\% confidence = 60.6\%]
    \item R9: ( Invest. -> Num.Pesq. ) 1ª [support = 42.9\% confidence = 87.8\%]
    \item R10: ( Num.Pesq. -> Patentes ) 2ª [support = 26.7\% confidence = 58.4\%]
    \item R11: ( Num.Pesq. -> Invest. ) 2ª [support = 38.4\% confidence = 84.0\%]
    \item R12: ( Num.Pesq. -> Fat.Export. ) 2ª [support = 26.9\% confidence = 58.7\%]
    %\item 25: ( EP -> R ) a [support = 20.9\% confidence = 79.4\%]
    %\item 26: ( EP -> I ) a [support = 20.9\% confidence = 79.4\%]
    %\item 27: ( ER -> P ) a [support = 20.9\% confidence = 69.9\%]
    %\item 28: ( ER -> I ) a [support = 25.3\% confidence = 84.7\%]
    %\item 29: ( EI -> P ) a [support = 20.9\% confidence = 72.4\%]
    %\item 30: ( EI -> R ) a [support = 25.3\% confidence = 87.6\%]
    %\item 31: ( EI -> P ) b [support = 19.7\% confidence = 63.7\%]
    %\item 32: ( EI -> R ) b [support = 23.8\% confidence = 76.9\%]
    %\item 33: ( EP -> I ) b [support = 19.7\% confidence = 74.8\%]
    %\item 34: ( EP -> R ) b [support = 16.5\% confidence = 62.6\%]
    %\item 35: ( ER -> I ) b [support = 23.8\% confidence = 88.6\%]
    %\item 36: ( ER -> P ) b [support = 16.5\% confidence = 61.4\%]
    %\item 37: ( IP -> E ) b [support = 19.7\% confidence = 67.8\%]
    %\item 38: ( IP -> R ) b [support = 23.1\% confidence = 79.5\%]
    \item R13: ( Invest.+Num.Pesq. -> Patentes ) 2ª [support = 23.1\% confidence = 60.2\%]
    \item R14: ( Invest.+Num.Pesq. -> Fat.Export. ) 2ª [support = 23.8\% confidence = 61.9\%]
    %\item 41: ( PR -> E ) a [support = 20.9\% confidence = 69.5\%]
    %\item 42: ( PR -> I ) a [support = 26.4\% confidence = 87.6\%]
    %\item 43: ( IP -> E ) a [support = 20.9\% confidence = 70.7\%]
    %\item 44: ( IP -> R ) a [support = 26.4\% confidence = 89.1\%]
    %\item 45: ( PR -> I ) b [support = 23.1\% confidence = 86.6\%]
    %\item 46: ( PR -> E ) b [support = 16.5\% confidence = 61.8\%]
    \item R15: ( Invest.+Num.Pesq. -> Fat.Export. ) 1ª [support = 25.3\% confidence = 59.1\%]
    \item R16: ( Invest.+Num.Pesq. -> Patentes ) 1ª [support = 26.4\% confidence = 61.5\%]
    %\item 49: ( EPR -> I ) a [support = 18.7\% confidence = 89.4\%]
    %\item 50: ( EIP -> R ) a [support = 18.7\% confidence = 89.4\%]
    \item R17: ( Fat.Export.+Invest.+Num.Pesq. -> Patentes ) 1ª [support = 18.7\% confidence = 73.8\%]
    %\item 52: ( EIP -> R ) b [support = 15.0\% confidence = 75.9\%]
    \item R18: ( Fat.Export.+Invest.+Num.Pesq. -> Patentes ) 2ª [support = 15.0\% confidence = 62.9\%]
    %\item 54: ( EPR -> I ) b [support = 15.0\% confidence = 90.7\%]
    \item R19: ( Invest.+Patentes+Num.Pesq. -> Fat.Export. ) 2ª [support = 15.0\% confidence = 64.7\%]
    \item R20: ( Invest.+Patentes+Num.Pesq. -> Fat.Export. ) 1ª [support = 18.7\% confidence = 71.0\%]
\end{itemize}

Essas relaçoes descrevem melhor as consequências do perfil apresentado pelas regras BACAR, fundamental para se concluir como a combinação das variáveis influencia nas chances de uma empresa ter se mantido ativa durante todo o período analisado.

Entende-se que emissão de patentes e o faturamento com exportações são consequências de um bom desempenho comercial da empresa. Por isso, apenas regras BCAAR que estabelecem relações entre investimento e pessoal dedicado em pesquisa e desenvolvimento numa determinada onda são enfatizadas nesta análise.

A título de exemplo, a Regra R9: \textit{( Invest. -> Num.Pesq. ) 1ª [support = 42.9\% confidence = 87.8\%]} mostra que 43\% das empresas que investiram a partir de 2.5\% de seu faturamento em pesquisa e desenvolvimento nos primeiros três anos do estudo também possuíam pesquisadores acima de 5\% de seus empregados, nesse mesmo período. Ambos os thresholds foram obtidos a partir do valor da mediana das variáveis. A confiança de 88\% apresentada pela regra mostra boa relação de um investimento incentivar outro.

A alta confiança para regras R3, R6, R9 e R11, que combinam investimento em pesquisa e desenvolvimento com número de empregados dedicados a pesquisa, mostram que há um consenso, neste contexto, de que realizar um destes investimentos implica em também realizar o outro.

Já ao observar o suporte e a confiança das regras R1, R2, R4, R5, R7, R8, R10 e R12, que combinam uma variável de investimento com uma variável de retorno, é notável que investimento em pesquisa e desenvolvimento mostrou melhor retorno em faturamento com exportações, enquanto dedicar funcionários a pesquisa o fez também com a emissão de patentes, mesmo com menos empresas se encaixando neste último cenário.

Observa-se ainda uma tendência de queda no suporte das regras R15 e R16 para as R13 e R14, que combinam ambas as variáveis de investimento com uma de retorno da primeira para a segunda ondas, refletindo os 10\% das empresas abordadas neste estudo que se tornaram inativas em algum momento do período analisado. No entanto, a estabilidade da confiança destas mesmas regras reflete uma potencial manutenção nas estratégias comerciais das empresas ao longo dos 7 anos do estudo.

Além disso, as regras R17, R18, R19, R20, apesar do baixo suporte, apresentam confiança significativa a partir de 60\%. Por exemplo, nas regras R17, R18, as poucas empresas que  tiveram investimento em pesquisa/desenvolvimento, possuem pesquisadores, e tiveram faturamento com exportação tiveram também emissão de patentes, porém com pequena diminuição no número de patentes entre as ondas.

Assim, pode-se observar que, neste contexto, investimentos em pesquisa e desenvolvimento e uma presença de pesquisadores entre os empregados, quando implantados em conjunto, contribuíram para favorecer as chances das empresas analisadas se manterem ativas ao longo dos 7 anos do estudo, refletindo sua atividade em emissão de patentes e faturamento com exportações.

\section{Conclusão}

Neste trabalho, foram analisadas as ações de investimentos em pesquisa e desenvolvimento com os retornos financeiros de empresas de tecnologia. Foram consideradas informações procurando obter relações entre Investimentos em pesquisa, Números de pesquisadores na empresa, Emissão de patentes, e Faturamento com exportações. A Análise permite avaliar as chances de uma empresa se manter ativa durante período longos analisando a efetividade do investimento em pesquisa da empresa.

Foi demostrado que a análise triádica é promissora na descrição de bases de dados longitudinais. O uso de ferramentas como o Lattice Miner combinado com uma preparação propícia das variáveis a serem analisadas, seus limiares e conclusões concisas acerca dos resultados obtidos, apresentam potencial para estudos em diversas áreas.

Espera-se que os resultados apresentados incentivem a expansão das aplicações da análise triádica para que possam produzir informação a serem consideradas na gestão das empresas.

\section*{\uppercase{Acknowledgements}}
The authors would like to thank: The National Council for Scientific and Technological Development of Brazil (CNPQ), The Coordination for the Improvement of Higher Education Personnel - Brazil (CAPES) (Grant
PROAP 88887.842889/2023-00 – PUC/MG, Grant
PDPG 88887.708960/2022-00 – PUC/MG - INFORMATICA and Finance Code 001), Minas Gerais State Research Support Foundation (FAPEMIG) under grant number APQ-01929-22, and the Pontifical Catholic University of Minas Gerais, Brazil.

% ARTICLE NEW SECTION
% \section{Call for Contributions}

% \textbf{IMPORTANT: TITLE, ABSTRACT AND KEYWORDS MUST BE WRITTEN IN ENGLISH}. Remaining sections of the paper may be written in english or portuguese.


% KDMiLe is organized alternately in conjunction with The Brazilian Conference on Intelligent Systems (BRACIS) and the Symposium on Database Systems (SBBD). The KDMiLe Program Committee invites submissions containing new ideas and proposals, and also applications, in the Data Mining and Machine Learning areas. Submitted papers will be reviewed based on originality, relevance, technical soundness and clarity of presentation. 

% % ARTICLE NEW SUBSECTION
% \subsection{Scope and Topics}

% KDMiLe welcomes articles on a full range of research and applications on data mining and machine learning, including (but not limited to):

% % LIST OF ITEMS

% \begin{enumerate}
% \item \textbf{Data Mining Topics}

% \begin{itemize}
% \item Association Rules
% \item Classification
% \item Clustering
% \item Data Mining Aplications
% \item Data Mining Foundations
% \item Evaluation Methodology in Data Mining
% \item Feature Selection and Dimensionality Reduction
% \item Graph Mining
% \item Massive Data Mining
% \item Multimedia Data Mining
% \item Multirelational Mining
% \item Outlier Detection
% \item Parallel and Distributed Data Mining
% \item Pre and Post Processing
% \item Ranking and Preference Mining
% \item Privacy and Security in Data Mining
% \item Quality and Interest Metrics
% \item Recommender Systems based on Data Mining
% \item Sequential Patterns
% \item Social Network Mining
% \item Stream Data Mining
% \item Text Mining
% \item Time-Series Analysis
% \item Visual Data Mining
% \item Web Mining
% \end{itemize}

% \item \textbf{Machine Learning Topics}

% \begin{itemize}
% \item Active Learning
% \item Bayesian Inference
% \item Case-Based Reasoning
% \item Cognitive Models of Learning
% \item Constructive Induction and Theory Revision
% \item Cost-Sensitive Learning
% \item Deep Learning
% \item Ensemble Methods
% \item Evaluation Methodology in Machine Learning
% \item Fuzzy Learning Systems
% \item Inductive Logic Programming and Relational Learning
% \item Kernel Methods
% \item Knowledge-Intensive Learning 
% \item Learning Theory
% \item Machine Learning Applications
% \item Meta-Learning
% \item Multi-Agent and Co-Operative Learning
% \item Natural Language Processing
% \item Online Learning
% \item Probabilistic and Statistical Methods
% \item Ranking and Preference Learning
% \item Recommender Systems based on Machine Learning
% \item Reinforcement Learning
% \item Semi-Supervised Learning
% \item Supervised Learning
% \item Self-Supervised Learning
% \item Unsupervised Learning
% \end{itemize}

% \end{enumerate}

% \subsection{Types of Submission}

% KDMiLe welcomes \textit{research} articles that both lay theoretical foundations and provide new insights into the aforementioned areas and also \textit{application} articles proposing to adapt existing data mining and machine learning technologies in some specific context, innovative commercial  implementations as well as reports and analysis of experiences in applying recent research advances in data mining and machine learning  to practical situations.  


% \subsection{Submission Instructions}

% Papers may be written in Portuguese or English, and must not exceed 8 pages. Papers exceeding this limit will be automatically rejected without being reviewed by the Program Committee. In addition, papers must be submitted in PDF format. Formats other than PDF will NOT be accepted.
% Papers exceeding this limit will be automatically rejected without being reviewed by the Program Committee. In addition, papers must be submitted in PDF format. Formats other than PDF will NOT be accepted. 



% \subsection{Format Instructions}


% This folder has the following files:

% \begin{itemize}
% 	\item kdmileb.bib - an example of bibliography entries
% 	\item kdmile.cls - the latex template class for kdmile
% 	\item kdmile.bst - the latex template for bibliography entries
% 	\item template-kdmile.tex - an example of tex file
% 	\item template-kdmile.pdf - an example of the output pdf file
% \end{itemize}

% Prospective authors should take a look at the \textit{template-kdmile.tex} for more information on the format, for example: how to add more than one institution to the affiliation and  references to books \cite{Baeza-YatesR99}, book chapters \cite{BorgidaCL09}, conference papers \cite{FerreiraGALV09}, conference papers on book series (e.g., LNCS) \cite{SilvaLC96}, journal articles \cite{LaenderMNM09}, PhD thesis \cite{Moro07} and information published online\footnote{Note: in case the URL refers to a software, dataset source or any other information that does not have a proper author, it is better to include it as a footnote, e.g.: ``Biblioteca Digital Brasileira de Computa\c{c}\~{a}o: http://www.lbd.dcc.ufmg.br/bdbcomp''} \cite{xpath}. The file \textit{kdmileb.bib} has examples of how to name conference proceedings and journals. In addition, Section 2 overviews the most important Latex instructions and is a ``must-read'' for authors with little or no experience in Latex, and Section 3 presents the most common errors.

% %%%%%%%%%%%%%%%%%%%%%%%%%%%%%%%%%%%%%%%%%%%%%%%%%%%%%%%%%%%%%%%%%%%%%%%%%%%%%%%%%%%%%%%%%%%%%%%

% \section{LATEX Instructions}

% This section overviews some basic instructions for writing an article to KDMiLe using LATEX.

% %%%%%%%%%%%%%%%%%%%%%%%%%%%%%%%
% \noindent{FIRST LINES}
% %%%%%%%%%%%%%%%%%%%%%%%%%%%%%%%

% The first lines of your tex file identify the document class, packages, new definitions and environments the article will use.
% Specifically:

% % Numbered list
% \begin{enumerate}
% 	\item The document class must be $kdmile.cls$ and is defined as follows:
	
% 				\begin{latexcode} 
% 		          $\backslash$documentclass[kdmile,a4paper]\{kdmile\}
% 		    \end{latexcode}
				
% 		\rmfamily Note that kdmile is published on A4 paper.

%   \item You may use as many packages as you want. However, we suggest you use $graphicx$ and $url$ for figures and url format, and $fontenc$ for avoiding warnings such as ``\textit{LaTeX Font Warning: Font shape 'OMS/cmtt/m/n' undefined}''. You can define them as follows:
    
%     	\begin{latexcode} 
% 				$\backslash$usepackage\{graphicx,url\} 
		
% 				$\backslash$usepackage[T1]\{fontenc\}\vspace{0.1in}
% 			\end{latexcode}
		
% 		Please note that including the package $cite$ may conflict with our template for bibliographic references ($kdmile.bst$).
	
% 	\item You may also use new definitions, such as the ones included in kdmile template:
	
	
% 			\begin{latexcode} 
% 		  $\backslash$newtheorem\{theorem\}\{Theorem\}[section] 
		  
% 			$\backslash$newtheorem\{conjecture\}[theorem]\{Conjecture\}
			
% 			$\backslash$newtheorem\{corollary\}[theorem]\{Corollary\}
			
% 			$\backslash$newtheorem\{proposition\}[theorem]\{Proposition\}
			
% 			$\backslash$newtheorem\{lemma\}[theorem]\{Lemma\}
			
% 			$\backslash$newdef\{definition\}[theorem]\{Definition\}
			
% 			$\backslash$newdef\{remark\}[theorem]\{Remark\}
% 			\end{latexcode}
			
% 	\item You may also add new environments, such as the one included in kdmile template:

% 		\begin{latexcode} 
% 				$\backslash$newenvironment\{latexcode\} 
				
% 		    \{$\backslash$vspace\{0.1in\}$\backslash$setlength\{$\backslash$parindent\}\{18pt\}\} 
		    
% 				\{$\backslash$vspace\{0.1in\}\}
% 		\end{latexcode}
		 		
% \end{enumerate}

% %%%%%%%%%%%%%%%%%%%%%%%%%%%%%%%
% \noindent{MANDATORY FIELDS}
% %%%%%%%%%%%%%%%%%%%%%%%%%%%%%%%

% The initial lines are followed by mandatory fields which compose the article metadata. It is very important that all of them be correctly written as follows.

% \begin{enumerate}

% 	\item Each article has a left and right-headers composed of two parts: the abbreviated names of the authors and the title. Both parts are defined as follows:
	
% 		\begin{latexcode} 
% 			$\backslash$markboth\{A. Plastino and S. de Amo and A. P. L. de Carvalho\}\{KDMiLe - Symposium on Knowledge Discovery,  Mining and Learning\} 
% 		\end{latexcode}
	
% 		Note that the first parameter has the authors, which are defined following these rules:
		
% \begin{itemize}
% 	\item If there is one author, then use author's full name, such as \textit{Alexandre Plastino}.
% 	\item If there are two or three authors, then abbreviate each author's first name such as \textit{A.Plastino and S. de Amo and A.P.L. de Carvalho}.
% 	\item If there are more than three authors, then the format is \textit{A. Plastino et. al.}.
% \end{itemize}
	
% 	The second parameter is the title. If the title is too long, contract it by omitting subtitles and phrases, \textit{not} by abbreviating words.
	
% 	\item The title of the article is defined as follows:
	
% 		\begin{latexcode}
% 				$\backslash$title\{KDMiLe - Symposium on Knowledge Discovery,  $\backslash\backslash$ Mining and Learning\}
% 		\end{latexcode}
	
% 		Note that if you want to break the title in two or more lines, just add a line within it with two consecutive backlashes ($\backslash\backslash$).
	
% 	\item After the title, you need to define the name of the authors and their affiliations, as follows.

% 		\begin{latexcode}
% 				$\backslash$author\{A. Plastino, V. Braganholo\}
				
% 				$\backslash$institute\{Universidade Federal Fluminense, Brazil $\backslash\backslash$
				
% 				$\backslash$email\{$\backslash$\{plastino,vanessa$\backslash$\}@ic.uff.br\}
				
% 		\end{latexcode}
		
% 	Note that both authors belong to the same university. If you have authors from different places, you write their information as follows:
		
% 		\begin{latexcode}		
% 				$\backslash$author\{S. de Amo$\backslash$inst\{1\}, A. P. L. de Carvalho$\backslash$inst\{2\}\}
				
% 				$\backslash$institute\{Universidade Federal de Uberl\^andia, Brazil $\backslash\backslash$
				
% 				$\backslash$email\{deamo@ufu.br\} 
				
% 				$\backslash$and
				
% 				$\backslash$institute\{Universidade de S\~ao Paulo, Brazil $\backslash\backslash$
				
% 				$\backslash$email\{andre@icmc.usp.br\} 
 
% 		\end{latexcode}

		
% 	\item The next step is to add the abstract within its environment, as follows.
	
% 		\begin{latexcode}		
% 				$\backslash$begin\{abstract\}
				
% Text of your abstract using from 100 to 300 words, without \textbf{any references}. 

% Also, do avoid breaking it into paragraphs.
				
% 				$\backslash$end\{abstract\}
% 		\end{latexcode}
					
% 	\item After the abstract, you should include information for the ACM Computing Classification.	
	
% 	The ACM Computing Classification System ---
% 	\url{https://www.acm.org/publications/class-2012} --- is a set of
% 	classifiers and concepts that describe the computing
% 	discipline. Authors can select entries from this classification
% 	system, via \url{https://dl.acm.org/ccs/ccs.cfm}, and generate the
% 	commands to be included in the \LaTeX\ source. \\

	

% 	\item Then, the keywords are specified in alphabetical order as follows.
	
% 	\begin{latexcode}
% 		$\backslash$keywords\{kdmile,template\}
% 	\end{latexcode}

% 	Note that keywords can also be expressions, such as \textit{Data Mining, Machine Learning}.

% 	\item The article then begins as follows.
	
% 	\begin{latexcode}
% 	   $\backslash$begin\{document\}
% 	\end{latexcode}
	
% 	\item It is common to thank funding agencies for research grants. If necessary, that information is defined in the \textit{bottomstuff} environment, as follows.
	
% 	\begin{latexcode}
% 		$\backslash$begin\{bottomstuff\}
	
% 		This work was partially funded by CNPq grant number XYZ. 
	
% 		$\backslash$end\{bottomstuff\}	
% 	\end{latexcode}
	
% 	\item All the previous items complete the initial part of the article. Then, you need to use the \textit{$\backslash$maketitle} command, which generates and formats all the previous items, as follows.
	
% 		\begin{latexcode}
% 	   $\backslash$maketitle
% 	\end{latexcode}
% \end{enumerate}

% %%%%%%%%%%%%%%%%%%%%%%
% \noindent{BODY CONTENT}
% %%%%%%%%%%%%%%%%%%%%%%

% The article then follows with its regular content with sections and subsections as follows.

% 	\begin{latexcode}
% 		$\backslash$section\{Section title\} \\
% 		\rmfamily which defines a section in first level (i.e., \textit{1. Section title}).
	
% 		\ttfamily$\backslash$subsection\{Subsection title\} \\
% 		\rmfamily which defines a section in second level (i.e., \textit{1.1. Subsection title}).
	
% 		\ttfamily$\backslash$subsubsection\{Subsubsection title\} 
% 		\rmfamily defines a section in third level (i.e., \textit{1.1.1. Subsubsection title}).
% 	\end{latexcode}

% 	For further information, we refer to the following web sources:
	
% 	\ttfamily
% \begin{itemize}
% 	\item http://en.wikibooks.org/wiki/LaTeX
% 	\item http://www.latex-project.org/
% \end{itemize}
	
% 	\rmfamily

% %%%%%%%%%%%%%%%%%%%%%%
% \noindent{FINISHING YOUR ARTICLE}
% %%%%%%%%%%%%%%%%%%%%%%

% At the end of your article, you may add appendix sections as follows.

% \begin{latexcode}
% 	$\backslash$appendix
	
% 	$\backslash$section\{Title of first appendix section\}

% 	Text text text
% \end{latexcode}

% After the appendix, you may also add a section for acknowledgments, as follows. Note that funding agencies should be mentioned as defined by the aforementioned environment $\backslash bottomstuff$.

% \begin{latexcode}
% 				$\backslash$begin\{ack\}
				
% 				Any acknowledgment you want.
				
% 				$\backslash$end\{ack\}
% \end{latexcode}

% Then, you add the dates (if you do not add these lines, the footer of the title page will be wrong), as follows:

% \begin{latexcode}
% 	$\backslash$begin\{received\}
% 	$\backslash$end\{received\}
% \end{latexcode}

% Finally, you must refer to the bibliography style and file, as follows.

% 	\begin{latexcode}
% 		$\backslash$bibliographystyle\{kdmile\}

% 		$\backslash$bibliography\{kdmileb\}
% 	\end{latexcode}

% \noindent where the file $kdmileb.bib$ has all references. We try to keep the names of proceedings and journals following the same pattern. All references must be defined in the bib file, as follows. 

% \begin{latexcode}
% @MISC\{xpath, \\\indent\indent
%   author = \{Anders Berglund and Scott Boag and Don Chamberlin and \\\indent\indent
%   Mary F. Fern$\backslash$'\{a\}ndez and Michael Kay  and Jonathan Robie and \\\indent\indent
%   J$\backslash$'\{e\}r$\backslash$\^\{o\}me Sim$\backslash$'\{e\}on\},\\\indent\indent
%   title = \{\{XML Path Language (XPath) 2.0, W3C Recommendation\}\},\\\indent\indent
%   howpublished = \{http://www.w3.org/TR/xpath20\},\\\indent\indent
%   year = \{2007\} \}\\

% @INCOLLECTION\{BorgidaCL09,\\\indent\indent
%   chapter = \{\{Logical Database Design: from Conceptual to Logical Schema\}\},\\\indent\indent
%   pages = \{1645-1649\},\\\indent\indent
%   title = \{\{Encyclopedia of Database Systems\}\},\\\indent\indent
%   publisher = \{Springer\},\\\indent\indent
%   year = \{2009\},\\\indent\indent
%   editor = \{Liu Ling and Tamer M. $\backslash$"\{O\}zsu\},\\\indent\indent
%   author = \{Alexander Borgida and Marco A. Casanova and Alberto H. F. Laender\},\\\indent\indent
%   address = \{Berlin\} \}\\

% @INPROCEEDINGS\{FerreiraGALV09,\\\indent\indent
%   author = \{Anderson A. Ferreira and Marcos Andr\{$\backslash$'e\} Gon$\backslash$c\{c\}alves and Jussara\\\indent\indent
% 	M. Almeida and Alberto H. F. Laender and Adriano Veloso\},\\\indent\indent
%   title = \{\{SyGAR - A Synthetic Data Generator for Evaluating Name Disambiguation\\\indent\indent
% 	Methods\}\},\\\indent\indent
%   booktitle = ecdl,\\\indent\indent
%   year = \{2009\},\\\indent\indent
%   pages = \{437-441\},\\\indent\indent
%   address = \{Corfu, Greece\} \}\\

% @ARTICLE\{LaenderMNM09,\\\indent\indent
%   author = \{Alberto H. F. Laender and Mirella M. Moro and Cristiano Nascimento\\\indent\indent
% 	and Patr$\backslash$'\{$\backslash$i\}cia Martins\},\\\indent\indent
%   title = \{\{An X-ray on Web-available XML Schemas\}\},\\\indent\indent
%   journal = sigmodrecord,\\\indent\indent
%   year = \{2009\},\\\indent\indent
%   volume = \{38\},\\\indent\indent
%   pages = \{37-42\},\\\indent\indent
%   number = \{1\} \}\\

% @PHDTHESIS\{Moro07,\\\indent\indent
%   author = \{Mirella M. Moro\},\\\indent\indent
%   title = \{\{The Role of Structural Aggregation for Query Processing over XML Data\}\},\\\indent\indent
%   school = \{University of California, Riverside (UCR)\},\\\indent\indent
%   year = \{2007\},\\\indent\indent
%   address = \{USA\} \}\\

% @INCOLLECTION\{SilvaLC96,\\\indent\indent
%   chapter = \{\{An Approach to Maintaining Optimized Relational Representations\\\indent\indent
% 	of Entity-Relationship Schemas\}\},\\\indent\indent
%   pages = \{292-308\},\\\indent\indent
%   title = \{Conceptual Modeling - ER'96\},\\\indent\indent
%   publisher = \{Springer\},\\\indent\indent
%   year = \{1996\},\\\indent\indent
%   editor = \{Bernhard Thalheim\},\\\indent\indent
%   author = \{Altigran Soares da Silva and Alberto H. F. Laender and Marco A. Casanova\},\\\indent\indent
%   volume = \{1920\},\\\indent\indent
%   series = \{Lecture Notes in Computer Science\},\\\indent\indent
%   booktitle = er \}
  
% \end{latexcode}

% %%%%%%%%%%%%%%%%%%%%5
% \section{Most Common Errors}

% This section lists the most common errors when writing your paper. Note that \textit{any} of these errors will certainly make the editors contact you for correction.

% \begin{itemize}
% 	\item Files do not compile. Editors will create the pdf files from the tex sources you submitted.
% 	\item Refering to work as ``paper'': you should refer to your work as article.
% 	\item kdmile metadata different from pdf file: author names must be exactly the same in the pdf as in the metadata at kdmile website [including abbreviations and uppercase letters].
% 	\item Generic CCS Concepts: in the item ``CCS Concepts'', the more specific the concept the better. Identify the lowest branches of the tree that seem to apply to your particular paper.	
% 	\item Wrong header: with two or three authors, the header is as ``A. Plastino and S. de Amo and A. P. L. de Carvalho''; otherwise, it is ``A. Plastino et. al.'' (use $\backslash markboth$).
% 	\item Acknowledgments as section/subsection: acknowledgments must be in the appropriate place (using environment $\backslash bottomstuff$).
% 	\item Long/wrong abstract: abstract should have only one paragraph; in the kdmile metadata, it cannot have tex commands.
% 	\item Incomplete/wrong references: *all* references must strictly follow kdmile template, i.e., conference/workshop papers with complete authors, title, proceedings name, year, paper initial and final pages, and city/country address (no series, no url, no date, no total number of pages, no publisher - see template example); journal articles must have authors, title, journal name, pages, both volume and number; double check the template for conference papers published on Lecture Notes; webpage can be footnote instead of reference.
% 	\item Undefined references.
% 	\item Style: try to avoid orphan words (one word alone in one line), orphan lines (one line alone at the beginning of a page), single subsections (e.g., section 3, subsection 3.1, section 4); place figures and tables at the top \url{[t]} or bottom \url{[b]} of your pages.
% 	\item File compiles with warnings: try to eliminate all warnings, they may affect the final pdf.
% 	\item Wrong characters: words in Portuguese (even inside bib file) must have proper characters (e.g., Computa$\backslash$c\{c\}$\backslash$\textasciitilde\{a\}o).
% 	\item Outdated template: check the website for the updated template (which corrected some errors from the previous one).
% \end{itemize}
	
% INCLUDE BIBLIOGRAPHY WHICH MUST FOLLOW kdmile.bst TEMPLATE
\bibliographystyle{kdmile}
\bibliography{kdmileb}
% For information on how to write bibliography entries, 
% see file kdmileb.bib

\begin{received}
\end{received}

\end{document}
